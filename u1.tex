\documentclass[10pt,ngerman]{article}
\usepackage[ngerman]{babel}
\usepackage{listings}
\usepackage[utf8]{inputenc}
\usepackage{listings}
\usepackage{color}
\usepackage{amsthm}
\usepackage{amssymb}
\usepackage{amsmath}
\usepackage{tikz}


\usepackage{graphicx}
\usepackage[top=1.2in, left=1.2in]{geometry}

\begin{document}
	\title{AlP3 - Übungsblatt 1}
	\author{León Dirmeier}
	\maketitle
	
	\renewcommand\thesubsection{(\alph{subsection})}
	\definecolor{lightgreen}{rgb}{0.41,0.98,0.67}
	\definecolor{codegray}{rgb}{0.5,0.5,0.5}
	\lstdefinestyle{code}{
		numbers=right,
		numberstyle=\tiny\color{blue},
		keywordstyle=\color{magenta},
		stringstyle=\color{red},
		%showspaces=true,
		backgroundcolor=\color{lightgreen},
		breaklines=true               
	}
	%\pagecolor{magenta}
	\lstset{style=code}

	\section{Rekursionsgleichungen}
	\subsection{}
	Rekursive Form: $T(1)=0$ und $T(n)=\lceil T(\frac{n}{2}) \rceil + \lfloor T(\frac{n}{2})\rfloor + n$\\
	Geschlossene Form: $T(n)=n*log_2(n)$\\\\
	Die Rundungsklammern können also außer Acht gelassen werden.
	Angenommen $\forall n | n=2^i$, dann ist $(n+1) = 2*n$ \\
	Die Rundungsklammern können also außer Acht gelassen werden.\\\\
	\begin{proof}
	
	Induktionsanfang:\\
	$T(1)=0$ und $T(1)=1*log_2(1)=0$\\\\
	Induktionsschritt:\\
	$T(n+1)=T(\frac{n+1}{2})+T(\frac{n+1}{2})+(n+1)$ und $T(n+1)=(n+1)log_2(n+1)$\\
	$T(n+1)=T(n)+T(n)+(n+1)$ und $2n*log_2(2n)$\\
	$T(n+1)=T(\frac{n+1}{2})+T(\frac{n+1}{2})+(n+1)$ und $2n*log_2(n)+2n*log_2(2)$\\
	$T(n+1)=T(\frac{n+1}{2})+T(\frac{n+1}{2})+(n+1)$ und $2n*log_2(n)+(n+1)$\\\\
	Einsetzen der Induktionsvoraussetzung $IV$:\\
	$T(n+1)=2*IV+(n+1)$ und $T(n+1)=2*IV+(n+1)$\\	
	\end{proof}
	
	\subsection{}
	Rekursive Form: $S(1)=1$ und $S(n)=\sum_{i=1}^{n-1}i*S(i)$\\
	Geschlossene Form: $\lceil \frac{n!}{2}\rceil$
	\\Die Klammern sind wegen S(1) und S(2) nötig.
	\\\\
	\begin{proof}
	
	Induktionsanfang:\\
		$S(1)=1$ und $S(1)=\lceil \frac{1!}{2} \rceil = \lceil \frac{1}{2} \rceil = 1$\\	
	 $S(2)=\sum_{i=1}^{1}i*S(1)=1*S(1)=1$ und $S(2)=\lceil \frac{2!}{2} \rceil=1$\\\\
	 Induktionsschritt:\\
	 $S(n+1)=\sum_{i=1}^{n}i*S(i)$ und $S(n+1)=\lceil \frac{(n+1)!}{2}\rceil$\\\\
	 Die Aufrundungsklammern werden ab jetzt der Einfachheit halber weggelassen, da sie nur für den S(1)-Fall wichtig sind.\\
	 $S(n+1)=\sum_{i=1}^{n-1}i*S(i)+\sum_{i=n}^{n}i*S(i)$ und $S(n+1)=(n+1)* \frac{n!}{2}$\\
	 $S(n+1)=\sum_{i=1}^{n-1}i*S(i)+n*S(n)$ und $S(n+1)=1* \frac{n!}{2}+n* \frac{n!}{2}$\\\\
	 Einsetzen der Induktionsvoraussetzung $IV$:\\
	 $S(n+1)=IV+n*S(n)$ und $S(n+1)=IV+n*\frac{n!}{2}=IV+n*S(n)$
	\end{proof}
	\section{Die Ungleichung vom arithmetischen und geometrischen Mittel}
	
	\subsection{}
	\begin{proof}
	\[P(2)=x_1x_2 \leq (\frac{x_1+x_2}{2})^2\]
	\[ \Longleftrightarrow4x_1x_2 \leq (x_1+x_2)^2\]
	\[ \Longleftrightarrow4x_1x_2 \leq x_1^2-2x_1x_2+x_2^2\]
	\[ \Longleftrightarrow2x_1x_2 \leq x_1^2+x_2^2\]
	\[\Longleftrightarrow0 \leq x_1^2-2x_1x_2+x_2^2\]
	\[\Longleftrightarrow0 \leq (x_1-x_2)^2\]
\end{proof}
	\subsection{}
	Wir können $x_n$ so wählen, dass der Wert der Klammer sich nicht ändert:
	$$x_n=\frac{x_1+...+x_{n-1}}{n-1}$$\\
	\begin{proof}
	$$\frac{x_1+...+x_{n-1}}{n-1}=\frac{x_1+...+x_{n-1}+(\frac{x1+...+x_{n-1}}{n-1})}{n}$$\\
	$$\frac{x_1+...+x_{n-1}}{n-1}=\frac{x1+...+x_{n-1}}{n}+\frac{x_1+...+x_{n-1}}{n(n-1)}$$\\
	$$n*(x_1+...+x_{n-1})=(n-1)*(x_1+...+x_{n-1})+(x_1+...+x_{n-1})$$\\
	$$n*(x_1+...+x_{n-1})=n*(x_1+...+x_{n-1})$$
	\end{proof}
	Nun der Beweis, dass \(P(n)\implies P(n-1)\)\\
	\begin{proof}
	\[P(n)=x_1...x_n \leq \frac{x_1+...+x_n}{n}^n\]
	Den äquivalenten Term einsetzen:\\
	\[P(n)=x_1...x_n \leq\frac{x_1+...+x_{n-1}}{n-1}^n\]\\
	\[x_1...x_{n-1}x_n \leq \frac{x_1+...+x_{n-1}}{n-1}^{n-1}*\frac{x_1+...+x_{n-1}}{n-1}$$\\
	$$x_1..x_{n-1} \leq \frac{x_1+...+x_{n-1}}{n-1}^{n-1}\] 
	\end{proof}
%	\subsection{}
%	\begin{proof}
%		\[P(2n)=P(n)P(n)\leq \frac{x_1+...+x_n+x_1+...+x_n}{2n}^{2n}\]
%		\[P(n)P(n)\leq \frac{2(x_1+...+x_n}{2n})^{2n} \]
%		\[P(n)P(n)\leq \frac{x_1+...+x_n}{n}^{2n} \]
%		\[P(n)P(n)\leq \frac{x_1+...+x_n}{n}^n*\frac{x_1+...+x_n}{n}^n\]				
%	\end{proof}
	\subsection{}
	\begin{proof}
		\[P(2n)=(x_1...x_n)(x_{n+1}...x_{2n}) \leq \frac{x_1+...+x_{2n}}{2n}^{2n}\]
		da gilt: \[(x_1..x_n) \leq \frac{x_1+...+x_n}{n}^n\]
		setzen wir: \[(x_1...x_n)(x_{n+1}...x_{2n}) \leq \frac{x_1+...+x_n}{n}^n * \frac{x_{n+1}+...+x_{2n}}{n}^n\]
		\[(x_1...x_n)(x_{n+1}...x_{2n}) \leq \frac{(x_1+...+x_n)(x_{n+1}+...+x_{2n})}{n^2}^n\]
		aus P(2) folgt: \[(x_1+...+x_n)(x_{n+1}+...+x_{2n}) \leq \frac{x_1+...+x_{2n}}{2}^2 \]
		dies wird eingesetzt in die Formel: \[(x_1...x_n)(x_{n+1}...x_{2n}) \leq(\dfrac{(\frac{x_1+...+x_{2n}}{2})^2}{n})^n \]
		\[(x_1...x_n)(x_{n+1}...x_{2n}) \leq(\dfrac{(x_1+...+x{2n})^2}{n2^2})^n \]
		\[(x_1...x_n)(x_{n+1}...x_{2n}) \leq \dfrac{x_1+...+x_{2n}}{2n}^{2n} \]
		
	\end{proof}
	\subsection{}
	\begin{proof}
		Induktionsanfang: \[ P(1)=x_1 \leq \frac{x_1}{1}^1\]
		\[P(1)=x_1 \leq x_1 \]
		P(2) wurde in (b) bewiesen.\\
		Für jedes n gibt es entweder eine Darstellung P(2k) oder P(2k+1).\\
		Für P(2k+1) wenden wir (a) an und wandeln es in P(2k) um.\\
		Nun wenden wir (c) Solange auf P(2n) an, bis 2k=2 ist.\\
		Somit sind wir bei unserem Anker P(2) angekommen.
		\end{proof}
	\section{Manipulation elementarer Funktionen}
	\[log(a^nb^n)=^{Produktregel}log(a)^n+log(b)^n=^{Potenzregel}n(log(a)+log(b))\]
	\[b^{n(log(a)})=^{Basiswechsel}a^{n(log(b))}=^{Potenzregel}a^{log(b^n)}\]
	\[\sqrt[b]{\frac{a^n}{b^m}}=^{Quotientenregel}\sqrt[b]{a^{n-m}}=^{Wurzelregel}a^{\frac{n-m}{b}}\]
	\[log_a(n^{log_ba})=^{Potenzregel}log_b(a)log_a(n)=^{Basiswechsel}\frac{log_ka}{log_kb}*\frac{log_kn}{log_ka}=\frac{log_kn}{log_kb}=^{Basiswechsel}log_bn\]
		
\end{document}	